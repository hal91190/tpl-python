\documentclass[presentation]{beamer}
\usepackage[T1]{fontenc}
\usepackage[french]{babel}
\usepackage{amsmath,amssymb}
\usepackage{graphicx}
\usepackage{geometry}
\usepackage{color,float,epsf,caption,subcaption}
\usepackage{pslatex}
\usepackage{amsthm}
\usepackage{amsfonts}
\usepackage{dsfont}
\usepackage{mathrsfs}
\usepackage{tikz}
\usepackage{graphicx}
\usepackage{appendix}
\usepackage{multicol}
\usepackage{pifont}
\usepackage{bbding}
\usepackage{listings}
\lstloadlanguages{bash}


\usetheme{default}
\usecolortheme{dolphin}
\usefonttheme{structuresmallcapsserif}
\beamertemplatenavigationsymbolsempty

\defbeamertemplate{footline}{centered page number}
{%
	\vspace*{-1cm}
  \hspace*{\fill}%
  \usebeamercolor[fg]{page number in head/foot}%
  \usebeamerfont{page number in head/foot}%
  \insertframenumber\,/\,\inserttotalframenumber%
  \hspace*{\fill}\vskip2pt%
}


\definecolor{vertFonce}{RGB}{8,112,33}
\definecolor{turquoise}{rgb}{0.3450, 0.63137, 0.5372}
\definecolor{bleuClair}{rgb}{0.29803, 0.63137, 0.67843}
\definecolor{gris}{rgb}{0.5470, 0.57058, 0.60588}
\definecolor{rouge}{rgb}{0.64313, 0.14117, 0.11764}
\definecolor{yelow}{rgb}{0.976470, 0.898039 , 0.392156863}
\definecolor{applegreen}{rgb}{0.55, 0.71, 0.0}
\definecolor{bostonuniversityred}{rgb}{0.8, 0.0, 0.0}
\definecolor{ao}{rgb}{0.0, 0.5, 0.0}

\newcommand{\valid}{\textcolor{ao}{\Checkmark}}
\newcommand{\unvalid}{\textcolor{bostonuniversityred}{\XSolidBrush}}

\title{Git}
\subtitle{Logiciel de Gestion de Versions}
\author{Yann Rotella}
\institute{IN100, Université de Versailles Saint Quentin en Yvelines}
\date{Décembre 2020}



\titlegraphic{
\includegraphics[height=2.5cm]{logo_uvsq}
%\qquad
%\includegraphics[height=1.7cm]{figures/inria}
}

\begin{document}
\lstset{language=bash}
\lstset{basicstyle=\tt\bfseries\color{green},backgroundcolor=\color{black}}
\begin{frame}[noframenumbering]
\titlepage
\end{frame}
\setbeamertemplate{footline}[centered page number]

\begin{frame}{Travailler en groupe}
On veut éviter les problèmes suivant:
\begin{itemize}
	\item Modifications simultanées;
	\item Bugs: "hier \c ca marchait, maintenant j'ai des bugs partout";
	\item Travailler hors connexion ou avec une mauvaise connexion.
\end{itemize}
\pause
Une bonne solution:
\begin{center}
	\textbf{Un logiciel de gestion de versions}
\end{center}
\end{frame}

\begin{frame}{Qu'est ce que \c ca fait ?}
\begin{itemize}
	\item On garde en mémoire \textcolor{red}{\textbf{toutes}} les anciennes versions (historique).
	\item \textcolor{red}{\textbf{Qui}} a modifié \textcolor{red}{\textbf{quoi}} et \textcolor{red}{\textbf{pourquoi}} ?
	\item Possibilité de \textcolor{red}{\textbf{fusionner}} intelligemment un travail simultané.
\end{itemize}
\pause
\hspace{1cm}
\begin{center}
	On utilisera donc \textcolor{blue}{Git} pour \textcolor{blue}{travailler à plusieurs} et/ou \textcolor{blue}{un projet long}.
\end{center}
\end{frame}

\begin{frame}{Choix du logiciel de gestion de versions}
\begin{multicols}{2}

\textbf{Centralisé} (SVN):
\begin{itemize}
	\item Un serveur conserve tout et chaque utilisateur s'y connecte.
\end{itemize}

\textbf{Décentralisé} (Mercurial, \textcolor{blue}{Git}):
\begin{itemize}
	\item Chacun possède l'historique
	\item Fonctionnement pair à pair
\end{itemize}

\begin{figure}%
\begin{tikzpicture}
	\node at (0,0) {\includegraphics[height = 1.4cm]{server_free_of_rights}};
	\draw[thick,<->,>=latex] (-1,1.5) -- (-.5,.5);
	\draw[thick,<->,>=latex] (-1.5,-1) -- (-.5,-.5);
	\draw[thick,<->,>=latex] (1.2,1.5) -- (.4,.6);
	\draw[thick,<->,>=latex] (2,1.5) -- (.5,.6);
\end{tikzpicture}
\end{figure}

\begin{figure}%
\begin{tikzpicture}[scale=.7]
	\draw[thick,<->,>=latex] (0,0) -- (-1,2);
	\draw[thick,<->,>=latex] (0,0) -- (3,1);
	\draw[thick,<->,>=latex] (0,0) -- (2,2.5);
	\draw[thick,<->,>=latex] (3,1) -- (2,2.5);
	\draw[thick,<->,>=latex] (2,2.5) -- (3,3.5);
\end{tikzpicture}
\end{figure}

\end{multicols}
\end{frame}

\begin{frame}{Choix du logiciel de gestion de versions}
\begin{multicols}{2}
\begin{figure}%
\begin{tikzpicture}[scale=1.2]
	\node at (0,0) {\includegraphics[height = 1.4cm]{server_free_of_rights}};
	\draw[thick,<->,>=latex] (-1,1.5) -- (-.5,.5);
	\draw[thick,<->,>=latex] (-1.5,-1) -- (-.5,-.5);
	\draw[thick,<->,>=latex] (1.2,1.5) -- (.4,.6);
	\draw[thick,<->,>=latex] (2,1.5) -- (.5,.6);
	\node[below] at (0,-.6) {GitHub};
\end{tikzpicture}
\end{figure}
\begin{itemize}
\item[\valid] Rapide
\item[\valid] Branches (plus tard)
\item[\unvalid] Un peu complexe
\item[\unvalid] Pas d'interface graphique
\item[\valid \valid] GitHub
\item[\valid \valid] Grande communauté
\end{itemize}
\end{multicols}
\end{frame}

\begin{frame}[fragile]{Installation de Git}
Normalement déjà fait pour récupérer les TD et le cours
\begin{itemize}
	\item \textbf{Linux}:
	\begin{lstlisting}
$ sudo apt-get install git-core gitk 
\end{lstlisting}
	\item \textbf{Windows}: msysgit.github.io
	\item \textbf{MAC OS}: git-osx-installer (par exemple)
\end{itemize}
\pause
\begin{center}
	Dans tous les cas, on utilisera la console. C'est l'occasion de se familiariser avec le bash !
\end{center}
\end{frame}

\begin{frame}[fragile]{Configuration}
\begin{lstlisting}
$ git config --global color.diff auto
$ git config --global color.status auto
$ git config --global color.branch auto
$ git config --global user.name "mon_pseudo"
$ git config --global user.email my_email
\end{lstlisting}
\end{frame}

\begin{frame}[fragile]{Naviguer dans le terminal}
\begin{lstlisting}
$ cd
$ ls
$ ls -a
$ mkdir
$ cd ..
$ git init
\end{lstlisting}
\end{frame}

\begin{frame}[fragile]{Cloner un dépôt Git}
Dans le dossier où on veut travailler:
\begin{lstlisting}
$ git clone https://github.com/depot_git
\end{lstlisting}
Cette solution demandera le nom d'utilisateur/ mot de passe à chaque fois.
\pause

\textbf{Utiliser SSH (avancé):}
\begin{lstlisting}
$ git clone git@github.com:depot_git
\end{lstlisting}
Cette solution nécessite de créer des clefs SSH (RSA).
\begin{figure}%
\begin{tikzpicture}
	\node at (0,0) {\includegraphics[height=2cm]{folder_free_of_rights}};
	\node at (0,-1.2) {.git};
	\node at (4,0) {\includegraphics[height=2cm]{fichier_free_of_rights}};
	\node at (4,-1.2) {fichier1.py};
	\node at (8,0) {\includegraphics[height=2cm]{fichier_free_of_rights}};
	\node at (8,-1.2) {fichier2.py};
\end{tikzpicture}
\end{figure}
\end{frame}

\begin{frame}[fragile]{Les informations du dépôt}
Connaître les fichiers modifiés:
\begin{lstlisting}
$ git status
\end{lstlisting}
Connaître les modifications du ou des fichiers:
\begin{lstlisting}
$ git diff nom_fichier
$ git diff
\end{lstlisting}

\end{frame}

\begin{frame}[fragile]{Les commits}
\begin{itemize}
\item Les commits servent à modifier les fichiers au sein du dépôt git, afin de garder l'historique.
\end{itemize}
\begin{lstlisting}
$ git add fichier1.py fichier2.py
$ git commit -a -m 'mon message'
$ git commit fichier1.py -m 'mom message'
\end{lstlisting}

\pause
\begin{itemize}
\item Les commits modifient \textbf{\textcolor{blue}{localement}} le dépôt.
\end{itemize}
\begin{figure}%
\begin{tikzpicture}
	\node at (2,0) {\includegraphics[height=2cm]{pc_free_of_rights}};
	\node at (6,2) {\includegraphics[height=2cm]{server_free_of_rights}};
	\draw[<->,thick,>=latex] (3.1,.5) -- node {\textcolor{red}{\XSolid}} (5.1,1.5);
	\node at (6,0.5) {Serveur};
	\node at (0,0) {\includegraphics[height=1cm]{fichier_free_of_rights}};
	\node at (0,1.5) {\includegraphics[height=1cm]{fichier_free_of_rights}};
	\draw[->,very thick,blue] (0,1.2) -- (0,.3);
	\draw (0,0.75) node[left] {\texttt{git commit}};
\end{tikzpicture}
\end{figure}
\end{frame}
\begin{frame}[fragile]{Naviguer dans l'historique}
\begin{lstlisting}
$ git log
$ git log --stat
$ git log -p
\end{lstlisting}

À quoi ressemblent les log (stat exemple)
\pause
\begin{lstlisting}
commit cd7c6ddbf84d02a8e2ae6c47f9c10665de977ee9
Author: Yann Rotella <yann.rotella@uvsq.fr>
Date:   Sun Oct 25 19:07:56 2020 +0100

    mon message

 presentations/fichier.py | 31 +++++++++---
 1 file changed, 23 insertions(+), 
 8 deletions(-)
\end{lstlisting}

\pause
\textcolor{blue}{\textbf{Flèches directionnelles + PageUp / PageDown}}
\end{frame}

\begin{frame}[fragile]{Annuler les bêtises}
\begin{itemize}
	\item Annuler le dernier commit, mais ne modifie pas le fichier (soft):
\begin{lstlisting}
$ git reset HEAD^
\end{lstlisting}
\item Annuler et effacer les modifications (hard):
\begin{lstlisting}
$ git reset --hard HEAD^
\end{lstlisting}
\item Revenir au dernier commit:
\begin{lstlisting}
$ git checkout
\end{lstlisting}
\item Annuler le \texttt{git add}:
\begin{lstlisting}
$ git reset HEAD -- fichier_a_supprimer
\end{lstlisting}
\end{itemize}
\end{frame}

\begin{frame}[fragile]{Travailler avec les autres}
\begin{multicols}{2}
\begin{lstlisting}
$ git pull
\end{lstlisting}
\begin{figure}%
\begin{tikzpicture}
	\node at (0,0) {\includegraphics[height=2cm]{pc_free_of_rights}};
	\node at (0,4) {\includegraphics[height=2cm]{server_free_of_rights}};
	\draw[->,very thick] (0,2.8) -- (0,1.1);
\end{tikzpicture}
\end{figure}
\begin{itemize}
	\item[1]<2-3> Pas de changement depuis la dernière fois: aucun soucis (Fast-Forward).
	\item[2]<3> Vous avez fait des commits: les changements sont fusionnés intelligemment ou \textbf{il y a des conflits}, indiqués par <<<<<<< (cf TD)
\end{itemize}
\end{multicols}
\end{frame}

\begin{frame}[fragile]{Travailler avec les autres}
\begin{multicols}{2}
\begin{lstlisting}
$ git push
\end{lstlisting}
\begin{figure}%
\begin{tikzpicture}
	\node at (0,0) {\includegraphics[height=2cm]{pc_free_of_rights}};
	\node at (0,4) {\includegraphics[height=2cm]{server_free_of_rights}};
	\draw[<-,very thick] (0,2.8) -- (0,1.1);
\end{tikzpicture}
\end{figure}
\begin{itemize}
	\item[1]<2-3> Toujours de type Fast-Forward
	\item[2]<3> Personne ne doit avoir fait push, \textbf{\textcolor{blue}{on doit donc toujours pull avant de push}}
\end{itemize}
\end{multicols}
\end{frame}

\begin{frame}[fragile]{Annuler les commits qui ont été push}
\begin{itemize}
\item[1] Récupérer l'Identifiant du commit:
\begin{lstlisting}
$ git log
\end{lstlisting}
\item[2] Annuler le commit:
\begin{lstlisting}
$ git revert ID
\end{lstlisting}
\end{itemize}
pour l'Identifiant, il suffit de taper les 5-6 premiers caractères, et assurer l'unicité sur l'ensemble des commits.
\end{frame}

\begin{frame}[fragile]{Les branches (Git avancé)}
	On utilise les branches pour des sous-projets qui vont être longs.
	\begin{figure}%
	\begin{tikzpicture}
		\draw[fill = green] (0,0) rectangle node {\huge{branche1}} (5,1);
		\draw[fill = green] (5,-.2) -- (5.8,.5) -- (5,1.2) -- (5,-.2);
		
		\draw[fill = red] (-3,3) rectangle node {\huge{master}} (7,4);
		\draw[fill = red] (7,2.8) -- (7.8,3.5) -- (7,4.2) -- (7,2.8);
		
		\draw[->,very thick,blue,>=latex] (0,3.5) -- node[left] {\texttt{\bfseries{git branch}}} (0,.5);
		\draw[<-,very thick,blue,>=latex] (5,3.5) -- node[left] {\texttt{\bfseries{git merge}}} (5,.5);
	\end{tikzpicture}
	\end{figure}
\end{frame}

\begin{frame}[fragile]{Les branches (Git avancé)}
\begin{itemize}
\item Créer la branche:
\begin{lstlisting}
$ git branch nom_branche
\end{lstlisting}
\item Travailler sur la branche:
\begin{lstlisting}
$ git checkout nom_branche
\end{lstlisting}
\item Fusionner la branche avec master:
\begin{lstlisting}
$ git merge nom_branche
\end{lstlisting}
\item Supprimer la branche:
\begin{lstlisting}
$ git branch -d nom_branch
\end{lstlisting}
\end{itemize}
\end{frame}

\begin{frame}[fragile]{Autres}
Si on veut pull en urgence, stash permet de sauvegarder les modifications locales non-commitées.
\begin{lstlisting}
$ git stash
\end{lstlisting}
\pause
Très avancé: la gestion des branches vues précédemment peut devenir complexe, mais permet de couper un très gros projet en plein de branches, et d'intégrer les branches seulement quand le code est stable.
\end{frame}

\begin{frame}{Les bonnes pratiques}
\begin{itemize}
	\item<1-4> On commit fréquemment, mais uniquement du code qui \textbf{fonctionnne}.
	\item<2-4> On teste donc son code \textbf{avant} de commit.
	\item<3-4> On push rarement (à la fin de la journée / demi-journée de travail).
	\item<4> \textbf{On utilise Git !}
\end{itemize}
	
\end{frame}

\end{document}
